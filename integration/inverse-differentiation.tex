\chapter{The inverse of differentiation}
Differentiation is the process of taking a function (which must be of a certain type) and obtaining another function, which is called the \emph{derived function} or usually, the \emph{derivative} of the first function. The derivative is a function whose value at each point of the domain is equal to the gradient of the original function at the same point of the domain. Here the gradient is a geometric property of the graph of the function, which we define rigorously using a limit.

For a function to have a derivative, we first of all need both its domain and its co-domain to be $\R$ or subsets of $\R$. (Functions with other domains or co-domains can have derivatives, which mean something slightly different, but we will never look at any of these.) Secondly the function has to be continuous. Functions which are continuous at some points but not others can only have a derivative at those points where they are continuous. But these two conditions are not enough -- the function must also be `differentiable'. For reasonably straightforward functions of the type we are going to consider, to be differentiable at a certain point  means roughly that the function does not have `a corner' at that point.

Suppose we are given the derivative of a function. How can we find the original function? The first important point is that several different functions have the same derivative. This is because the derivative of any constant is zero, and therefore adding any constant to a function does not change its derivative. 
\[ \frac{d}{dx}(f + c) = \frac{d}{dx}f + \frac{d}{dx}c = f' + 0 = \frac{d}{dx}f \] 

This means that differentiation, a `mapping` which sends functions to other functions, is not an injection. We can find two functions, or even infinitely many functions, which all get mapped to the same function. This is a problem for finding an inverse. For a mapping to have an inverse, we need it to be both injective and surjective. 

We can take a sort of inverse of a non-injective function. If $f$ is a function which isn't injective, rather than define $f^{1-}(y)$ to be the single value $x$ such that $f(x) = y$, we say that $f^{-1}(y)$ is the set $\{ x : f(x) = y \}$.  In the case of the antiderivative, or the inverse of the derivative mapping, each of its values will be a set of functions, where any two functions differ by a constant.

We could choose some arbitrary member of that set, and call it `the' antiderivative. For example, there will always be exactly one member of the set which is equal to 0 at 0. However, in some cases this will not be the particular choice of antiderivative function which we think is natural, and sometimes there might be two or more which are natural for different reasons.  

For this sort-of-inversion of differentiation to work, we have to answer some more questions.
\begin{enumerate}
\item Is it possible for two functions, which do not differ by a constant, to have the same derivative?
\item How do we know whether a function is the derivative of something or not?
\item How can we work out what that thing is, if it exists?
\end{enumerate}

\subsection{Example}
We know that the derivative of $f(x) = x^2$ is $2x$. We also know that the derivative of $f(x) = x^2 + c$ is also $2x$, if $c$ is any real number. 

\section{Notes on the next part}
\begin{itemize}
\item Explain what the purpose of differentiation is
\item Point out that sometimes we care about the whole derivative function and sometimes we only want its value at certain points
\item Establish a link between these two roles for the derivative and the two ways of finding one -- geometric, algebraic.
\item See why the antiderivative at a single point does not mean much
\item The antiderivative function is useful sometimes, as is the difference between two values of the antiderivative.
\item Look at why the difference between two values is meaningful
\item By considering the derivative, try and find an interpretation for the difference between two values of the antiderivative.
\item Show the notation and the terminology involved for indefinite and definite integrals
\item Try and give some motivations for the antiderivative
\item Compare which motivations of the derivative have some crossover meaning and which don't.
\end{itemize}

\section{Things not in the right place yet}

 We justify this in several ways:
\begin{itemize}
\item The derivative of a function shows us how the value of that function changes as the input to the function changes. 
\item A very important case of the previous item is when we are looking at a function of time and the derivative indicates a rate of change.
\item Derivatives allow us to find local minima and maxima of a function (by looking for the points where the derivative is zero.)
\item We can use derivatives to estimate values of a function close to a point where we know the value of the function and of the derivative, as in Taylor expansions.
\end{itemize}

We will answer these questions in the same way we learnt about derivatives. We will use a quite technical theory involving geometry and limits to reason about functions and their rates of change. But once we have justified what we are doing, we will make calculations using symbolic methods which look much more like algebra than geometry, and don't explicitly involve limits.

\newglossaryentry{operator}
{
	name=operator,
	description={is a mapping that takes a function as input and returns another function as output.}
}
We can see differentiation as a special type of function -- one which takes a function as input and returns another, related function as output. We will call a function like this, which acts on functions themselves, rather than on points or numbers, an \newmention{operator}. Differentiation is an operator which sends a function to its derivative. 
