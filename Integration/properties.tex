\chapter{Properties of the antiderivative}

\newglossaryentry{antiderivative}
{
	name=antiderivative,
	description={The antiderivative of a function $f$ is a function whose derivative is $f$. If a function has an antiderivative, then there is a large collection of functions which are all antiderivatives.}
}
Suppose that we can take a function\footnote{In the following the term `function' always refers to a function from $\R$ to $\R$.} $f$ and by some method work out another function $F$, the \newmention{antiderivative} of $f$ which has the property that
\[ \frac{dF}{dx} = f \]
Here the equality means that the left and right hand side are equal at every point.
\[ \frac{dF}{dx} |_{x=k} = f(k) \]

If $f(x) = 0$ for all $x$, then $F$ might be any constant function. 
\[ \frac{d}{dx}c |_{x=k} = 0 = f(k) \]


